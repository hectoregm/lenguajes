\documentclass{article}
\usepackage[left=2cm,right=2cm,top=3cm,bottom=3cm,letterpaper]{geometry}
\usepackage[spanish]{babel}
\usepackage[utf8]{inputenc}
\author{Profesora: Karla Ramírez Pulido\\
  Ayudante teoría: Joshua Emmanuel Mendoza Mendieta\\
  Ayudante laboratorio: Héctor Enrique Gómez Morales}
\title{Practica 2 - Data Types Parte 1 de 2}
\date{Fecha de inicio: 6 de febrero de 2015\\
  \textbf{Fecha de entrega: 20 de febrero de 2015}}
\begin{document}
\maketitle
\section{Instrucciones}
En esta práctica se tienen once ejercicios, los primeros diez son
obligatorios siendo solamente el ultimo opcional con valor de un punto
extra. Por lo tanto la calificación máxima que se puede obtener en
esta práctica es 11.

Esta practica se puede entregar en equipos de a lo máximo dos
personas, pero se recomienda que esta practica la hagan de forma
individual.

Esta práctica debe ser implementada con la variante plai, es decir
su archivo rkt debe tener como primer linea lo siguiente:
\texttt{\#lang plai}

Todos los ejercicios requieren contar con pruebas mediante el uso de
la función \texttt{test}:
\begin{verbatim}
(test <result-expr> <expected-expr>)
\end{verbatim}

En donde \textit{result-expr} es la expresión que se evalúa a cierto
valor que es comparado con \textit{expected-expr} que es otra
expresión que evalúa al valor esperado. Si las dos expresiones evalúan
a lo mismo la prueba imprime el éxito de la prueba, en caso contrario
indicar un error.

\begin{verbatim}
> (test (+ 1 2) 3)
(good (+ 1 2) 3 3 "at line 34")

> (test (+ 1 2) 4)
(bad (+ 1 2) 3 4 "at line 36")
\end{verbatim}

Cada ejercicio debe contar al menos con cinco pruebas.

\section{Ejercicios}
\textbf{Sección I.} Define las funciones que se te piden. Puedes crear y utilizar
funciones auxiliares. No puedes utilizar funciones de Racket que
resuelvan \textbf{directamente} los ejercicios.
\begin{itemize}
\item $\textbf{mcd \& mcm}$ - Define las funciones mcd y mcm que
  toman dos números y regresan su máximo común divisor y mínimo común
  múltiplo respectivamente, \textit{i.e.}
\begin{verbatim}
> (mcd 72 32)
8
> (mcm 8 6)
24
\end{verbatim}

\newpage

\end{itemize}
\end{document}
