\documentclass{article}
\usepackage[left=2cm,right=2cm,top=3cm,bottom=3cm,letterpaper]{geometry}
\usepackage[spanish]{babel}
\usepackage[utf8]{inputenc}
\author{Profesora: Karla Ramírez Pulido \and
  Ayudante: Héctor Enrique Gómez Morales}
\title{Practica 2 - Tipos de Datos Parte 1 de 2}
\date{Fecha de inicio: 26 de agosto de 2015\\
  \textbf{Fecha de entrega: 9 de septiembre de 2015}}
\begin{document}
\maketitle
\section{Instrucciones}
En esta práctica se tienen dieciocho ejercicios.

Esta práctica debe ser implementada con la variante plai, es decir
su archivo con terminación \textit{.rkt} debe tener como primer linea lo siguiente:
\texttt{\#lang plai}. Pueden utilizar sólo la paquetería básica de
\texttt{racket/base}, las funciones que se implementaron en la
practica 1 y funciones auxiliares que ustedes definan.

Todos los ejercicios requieren contar con pruebas mediante el uso de
la función \texttt{test}:
\begin{verbatim}
(test <result-expr> <expected-expr>)
\end{verbatim}

En donde \textit{result-expr} es una expresión que se evalúa y su valor obtenido
es comparado con valor obtenido de la expresión \textit{expected-expr} que es el
valor esperado de la prueba. Si las dos expresiones evalúan a lo mismo la prueba
imprime el éxito de la prueba, en caso contrario indicar un error.

\begin{verbatim}
> (test (+ 1 2) 3)
(good (+ 1 2) 3 3 "at line 34")

> (test (+ 1 2) 4)
(bad (+ 1 2) 3 4 "at line 36")
\end{verbatim}

Cada ejercicio debe contar al menos con \textbf{cinco} pruebas.

\section{Ejercicios}
\textbf{Sección I. Tipos de datos recursivos y no recursivos} Define
los siguientes tipos de datos que se te piden.

\begin{itemize}
\item $\textbf{Array}$ - Definir un tipo de dato \texttt{Array}
  que tenga un constructor de tipo \texttt{MArray}. El entero sirve
  para definir el tamaño del arreglo., \textit{i.e.}
\begin{verbatim}
> (MArray 4 '(1 2 3))
(Marray 4 '(1 2 3))
\end{verbatim}

\item $\textbf{List}$ - Definir un tipo de dato recursivo llamado
  \texttt{MList} que tenga a la lista vacía \texttt{MEmpty} y el
  constructor de tipo \texttt{MCons}, \textit{i.e.}
\begin{verbatim}
> (MEmpty)
(MEmpty)
> (MCons 1 (MCons 2 (MCons 3 (MEmpty))))
(MCons 1 (MCons 2 (MCons 3 (MEmpty))))
> (MCons 7 (MCons 4 (MCons 10 (MEmpty))))
(MCons 7 (MCons 4 (MCons 10 (MEmpty))))
\end{verbatim}

\item $\textbf{NTree}$ - Definir un tipo de dato recursivo llamado
  \texttt{NTree} que tenga como una hoja nula \texttt{TLEmpty} y un
  constructor de tipo \texttt{NodeN} (estos árboles son n-ários), \textit{i.e.}
\begin{verbatim}
> (TLEmpty)
(TLEmpty)
> (NodeN 1 (list (TLEmpty) (TLEmpty) (TLEmpty)))
(NodeN 1 (list (TLEmpty) (TLEmpty) (TLEmpty)))
> (NodeN 1 (list (NodeN 2 (list (TLEmpty))) 
                 (NodeN 3 (list (TLEmpty))) 
                 (NodeN 4 (list (TLEmpty) (TLEmpty) (TLEmpty)))))
(NodeN
 1
 (list
  (NodeN 2 (list (TLEmpty)))
  (NodeN 3 (list (TLEmpty)))
  (NodeN 4 (list (TLEmpty) (TLEmpty) (TLEmpty)))))
\end{verbatim}

\item $\textbf{Position}$ - Define un tipo de dato llamado
  \texttt{Position} que tenga un constructor de tipo \texttt{2D-Point}
  que toma dos números reales que indican una posición
  en el plano cartesiano, \textit{i.e.}
\begin{verbatim}
> (2D-Point 0 0)
(2D-Point 0 0)
> (2D-Point 1 (sqrt 2))
(2D-Point 1 1.4142135623730951)
\end{verbatim}

\item $\textbf{Figure}$ - Define un tipo de dato llamado
  \texttt{Figure}
  que tenga tres constructores:
  \begin{itemize}
    \item{\texttt{Circle} - Un constructor que toma un centro dada por un
      \texttt{Position} y un radio.}
    \item{\texttt{Square} - Un constructor que toma una posición de la
      esquina superior izquierda del cuadrado y una longitud.}
    \item{\texttt{Rectangle} - Un constructor que toma una posición de
    la esquina superior izquierda del rectángulo, su ancho y su largo.}
  \end{itemize}
\begin{verbatim}
> (Circle (2D-Point 2 2) 2)
(Circle (2D-Point 2 2) 2)
> (Square (2D-Point 0 3) 3)
(Square (2D-Point 0 3) 3)
> (Rectangle (2D-Point 0 2) 2 3)
(Rectangle (2D-Point 0 2) 2 3)
\end{verbatim}


\textbf{Sección II. Funciones sobre Datos} Define
las siguientes funciones que se te piden. Puedes crear y utilizar
funciones auxiliares. No puedes utilizar funciones de Racket que
resuelvan \textbf{directamente} los ejercicios.

\item $\textbf{setvalueA}$ - Dado un arreglo de tipo \texttt{Array},
  una posición y un valor numérico v, regresar otro arreglo con el
  valor v intercambiado en la posición indicada del arreglo original.
  En caso de que la posición sea igual o mayor al tamaño especificado
  en el arreglo, regresar un error \texttt{Out of bounds},
  \textit{i.e.}
\begin{verbatim}
> (define ar (MArray 5 '(0 0 0 0 0)))
> ar
(MArray 5 '(0 0 0 0 0))
> (setvalueA ar 2 4)
(MArray 5 '(0 0 4 0 0))
> (setvalueA ar 4 4)
(MArray 5 ’(0 0 0 0 4))
> (setvalueA ar 5 4)
. . setvalueA: Out of bounds
\end{verbatim}

\item $\textbf{MArray2MList}$ - Dado un arreglo de tipo \texttt{MArray},
  regresar una lista de tipo \texttt{MList} que contenga todos los
  elementos del arreglo original, \textit{i.e.}
\begin{verbatim}
> (MArray2MList (MArray 0 '()))
(MEmpty)
> (MArray2MList (MArray 5 '("a" "b")))
(MCons "a" (MCons "b" (MEmpty)))
> (MArray2MList (MArray 3 '(1 2 3)))
(MCons 1 (MCons 2 (MCons 3 (MEmpty))))
\end{verbatim}

\item $\textbf{printML}$ - Dada una lista de tipo \texttt{MList},
  imprimirla en un formato legible. Puedes utilizar las funciones para
  manipular cadenas y la función \texttt{$\sim a$}, \textit{i.e.}
\begin{verbatim}
> (printML (MEmpty))
"[]"
> (printML (MCons 7 (MEmpty)))
"[7]"
> (printML (MCons 7 (MCons 4 (MEmpty))))
"[7, 4]"
> (printML (MCons (MCons 1 (MCons 2 (MEmpty))) (MCons 3 (MEmpty))))
"[[1, 2], 3]"
\end{verbatim}

\item $\textbf{concatML}$ - Dadas dos listas de tipo \texttt{MList},
  regresar la concatenación, \textit{i.e.}
\begin{verbatim}
> (concatML (MCons 7 (MCons 4 (MEmpty))) (MCons 1 (MEmpty)))
(MCons 7 (MCons 4 (MCons 1 (MEmpty))))
> (concatML (MCons 7 (MCons 4 (MEmpty))) (MCons 1 (MCons 10 (MEmpty))))
(MCons 7 (MCons 4 (MCons 1 (MCons 10 (MEmpty)))))
\end{verbatim}

\item $\textbf{lengthML}$ - Dada una lista de tipo MLista, regresar la
  cantidad de elementos que tiene, \textit{i.e.}
\begin{verbatim}
> (lengthML (MEmpty))
0
> (lengthML (MCons 7 (MCons 4 (MEmpty))))
2
\end{verbatim}

\item $\textbf{mapML}$ - Dada una lista de tipo MLista y una funcion de aridad 1,
  regresar una lista de tipo MLista con la aplicación de la funcion a cada uno de
  los elementos de la lista original, \textit{i.e.}
\begin{verbatim}
> (mapML add1 (MCons 7 (MCons 4 (MEmpty))))
(MCons 8 (MCons 5 (MEmpty)))
> (mapML (lambda (x) (* x x)) (MCons 10 (MCons 3 (MEmpty))))
(MCons 100 (MCons 9 (MEmpty)))
\end{verbatim}

\newpage

\item $\textbf{filterML}$ - Dada una lista de tipo MLista y un predicado de un
  argumento, regresar una lista de tipo MLista sin los elementos que al aplicar
  el predicado, regresa falso, \textit{i.e.}
\begin{verbatim}
> (filterML (lambda (x) (not (zero? x))) (MCons 2 (MCons 0 (MCons 1 (MEmpty)))))
(MCons 2 (MCons 1 (MEmpty)))
> (filterML (lambda (l) (not (MEmpty? l)))
            (MCons (MCons 1 (MCons 4 (MEmpty))) (MCons (MEmpty) (MCons 1 (MEmpty)))))
(MCons (MCons 1 (MCons 4 (MEmpty))) (MCons 1 (MEmpty)))
\end{verbatim}

Definamos los siguientes tipos de datos y valores:
\begin{verbatim}
(define-type Coordinates
  [GPS (lat number?)
       (long number?)])

(define-type Location
  [building (name string?)
            (loc GPS?)])

;; Coordenadas GPS
(define gps-satelite (GPS 19.510482 -99.23411900000002))
(define gps-ciencias (GPS 19.3239411016 -99.179806709))
(define gps-zocalo (GPS 19.432721893261117 -99.13332939147949))
(define gps-perisur (GPS 19.304135 -99.19001000000003))

(define plaza-satelite (building "Plaza Satelite" gps-satelite))
(define ciencias (building "Facultad de Ciencias" gps-ciencias))
(define zocalo (building "Zocalo" gps-zocalo))
(define plaza-perisur (building "Plaza Perisur" gps-perisur))

(define plazas (MCons plaza-satelite (MCons plaza-perisur (MEmpty))))
\end{verbatim}

\item $\textbf{haversine}$ - Dados dos valores de tipo \texttt{GPS} calcular
  su distancia usando la formula de \texttt{haversine}, \textit{i.e.}
\begin{verbatim}
> (haversine gps-ciencias gps-zocalo)
13.033219276117368
> (haversine gps-ciencias gps-perisur)
2.44727738966455
> (haversine gps-satelite gps-perisur)
23.391736010506026
\end{verbatim}

\newpage

\item $\textbf{gps-coordinates}$ - Dada una lista de edificios, regresar
  la lista de coordenadas GPS de los edificios, las listas deben ser construidas
  con el tipo de dato \texttt{MList}, \textit{i.e.}
\begin{verbatim}
> (gps-coordinates (MEmpty))
(MEmpty)
> (gps-coordinates plazas)
(MCons
 (GPS 19.510482 -99.23411900000002)
 (MCons (GPS 19.304135 -99.19001000000003) (MEmpty)))
\end{verbatim}

\item $\textbf{closest-building}$ - Dado \textit{b} un valor de tipo \texttt{building} y una lista de tipo \texttt{MList} de buildings, regresar el edificio mas cercano a \textit{b}, \textit{i.e.}
\begin{verbatim}
> (closest-building zocalo plazas)
(building "Plaza Satelite" (GPS 19.510482 -99.23411900000002))
> (closest-building ciencias plazas)
(building "Plaza Perisur" (GPS 19.304135 -99.19001000000003))
\end{verbatim}

\item $\textbf{buildings-at-distance}$ - Dado \textit{b} un valor de tipo \texttt{building}, una lista de tipo \texttt{MList} de buildings y una distancia \textit{d}, regresar una nueva lista de todos los edificios que están a distancia menor o igual a \textit{d} del edificio \textit{b}, \textit{i.e.}
\begin{verbatim}
> (buildings-at-distance ciencias plazas 10)
(MCons (building "Plaza Perisur" (GPS 19.304135 -99.19001000000003)) (MEmpty))
> (buildings-at-distance ciencias plazas 20)
(MCons (building "Plaza Perisur" (GPS 19.304135 -99.19001000000003)) (MEmpty))
> (buildings-at-distance ciencias plazas 25)
(MCons
 (building "Plaza Satelite" (GPS 19.510482 -99.23411900000002))
 (MCons (building "Plaza Perisur" (GPS 19.304135 -99.19001000000003)) (MEmpty)))
\end{verbatim}

\item $\textbf{area}$ - Dada una figura del tipo \texttt{Figure},
  regresar el área de la figura, \textit{i.e.}
\begin{verbatim}
> (area (Circle (2D-Point 5 5) 4))
50.26548245743669
> (area (Square (2D-Point 0 0) 20))
400
> (area (Rectangle (2D-Point 3 4) 5 10))
50
\end{verbatim}

\item $\textbf{in-figure?}$ - Dada una figura \texttt{fig} del tipo \texttt{Figure}
  y una posición \texttt{p} del tipo \texttt{2D-Point} regresa \#t si \texttt{p} esta
  dentro de \texttt{fig} y \#f en caso contrario, \textit{i.e.}
\begin{verbatim}
> (in-figure? (Square (2D-Point 5 5) 4) (2D-Point 6 6))
#t
> (in-figure? (Rectangle (2D-Point 5 5) 4 6) (2D-Point 4 4))
#f
\end{verbatim}

\end{itemize}
\end{document}
