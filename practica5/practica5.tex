\documentclass{article}
\usepackage[left=2cm,right=2cm,top=2cm,bottom=2cm,letterpaper]{geometry}
\usepackage[spanish]{babel}
\usepackage[utf8]{inputenc}

\usepackage{verbatim, array}
\usepackage{hyperref}
\usepackage{amsmath, amsfonts, amssymb}
\usepackage{graphicx}
\usepackage[T1]{fontenc}

\newcommand{\gradeone}{(\textbf{1pt}) }
\newcommand{\grade}[1]{(\textbf{#1pts}) }
\newcommand{\jimage}[2]{\includegraphics[width=#1\textwidth]{#2}\vskip10pt}
\newcommand{\jcimage}[2]{\begin{center}\includegraphics[width=#1\textwidth]{#2}\end{center}\vskip10pt}

\author{Profesora: Karla Ramírez Pulido \and
  Ayudante: Héctor Enrique Gómez Morales}
\title{Practica 5 - Interprete RCFAEL}
\date{Fecha de inicio: 14 de Octubre de 2015\\
  \textbf{Fecha de entrega: 4 de Noviembre de 2015}}
\begin{document}
\maketitle
\section{Instrucciones}

Para esta práctica, requerirás tomar como base el interprete de la practica anterior e implementar las características que se solicitan.

El objetivo de esta práctica es hacer un intérprete del lenguaje \texttt{RCFAEL} (Recursive, Conditional, Function, Arithmetic Expression and List) con \emph{cajas}.

Se tendrán dos sintaxis, la primera \texttt{RCFAELS} es una sintaxis que se define explícitamente las expresiones \texttt{with}, y la sintaxis \texttt{RCFAEL}.

En la anterior practica vimos que usar la aplicación de funciones para implementar la funcionalidad del \texttt{with}.

\begin{itemize}
\item \textbf{Recursive} El intérprete de nuestro lenguaje debe poder evaluar correctamente variables con expresiones que hacen autorreferencia en el mismo ambiente por medio de \emph{cajas}.
\item \textbf{Conditional} El intérprete evaluara correctamente el control de flujo \texttt{if} que toma expresiones que se evalúan a booleanos.
\item \textbf{FAE} El intérprete debe evaluar correctamente expresiones de la práctica pasada como son funciones, aplicaciones y operaciones binarias predefinidas así como su extensión a operaciones unarias predefinidas y la inclusión de valores booleanos.
\item \textbf{List} El interprete debe evaluar correctamente expresiones para la construcción de listas y sus operaciones unarias correspondientes.
\end{itemize}


\begin{tabular}{c c}
  \begin{minipage}[t]{10cm}
\begin{verbatim}
<RCFAEL>::= <id>
          | <num>
          | <bool>
          | <MList>
          | {with {{<id> <RCFAEL>}+} <RCFAEL>}
          | {rec {{<id> <RCFAEL>}+} <RCFAEL>}
          | {fun {<id>*} <RCFAEL>}
          | {if <RCFAEL> <RCFAEL> <RCFAEL>}
          | {equal? <RCFAEL> <RCFAEL>}
          | {<op> <RCFAEL>}
          | {<binop> <RCFAEL> <RCFAEL>}
          | {<RCFAEL> <RCFAEL>*}

<id>::= a|...|z|A|...|Z|aa|ab|...|aaa|...
        (Cualquier combinación de caracteres alfanuméricos
         con al menos uno alfabético)

<num>::= ...|-2|-1|0|1|2|...

<bool>::= true | false

<MList>::= {empty}
         | {cons <RCFAEL> <RCFAEL>}

\end{verbatim}
  \end{minipage}
  &
  \begin{minipage}[t]{6cm}
\begin{verbatim}
   <op>::= inc
         | dec
         | zero?
         | num?
         | neg
         | bool?
         | first
         | rest
         | empty?
         | list?

<binop>::= +
         | -
         | *
         | /
         | <
         | >
         | <=
         | >=
         | and
         | or

\end{verbatim}
  \end{minipage}
\end{tabular}

Esta práctica debe ser implementada con la variante plai, es decir
su archivo con terminación \textit{.rkt} debe tener como primer linea lo siguiente:
\texttt{\#lang plai}.

Todos los ejercicios requieren contar con pruebas mediante el uso de
la función \texttt{test}.

\section{Ejercicios}

\begin{enumerate}

\item \grade{1} \textbf{Booleanos} Agrega al tipo RCFAEL y RCFAEL-Value la variante booleana. Esto es que en sintaxis concreta puedas reconocer símbolos \texttt{true} y \texttt{false} y al momento de procesar la lista de símbolos de una expresión que los incluya con \emph{parse}, en el árbol de sintaxis abstracta los manejes por medio de la variante de tipo \texttt{bool} (análogo a \texttt{num}), adicional a la variante \texttt{boolV} para el intérprete. \\

\item \grade{1} \textbf{Listas} Agrega al tipo RCFAEL y RCFAEL-Value los constructores específicos para la creación y manejo de listas. En un README aparte explica por qué es necesario tener los variantes de tipo de lista para los dos tipos de datos RCFWAEL y RCFWAEL-Value por separado. \\

\item \grade{1} \textbf{if} Agrega al parser y al intérprete la variante \texttt{if}, que recibe tres argumentos que son expresiones RCFAEL. Al interpretarse, si la evaluación del primer argumento es \texttt{(boolV \#t)}, la evaluación del \texttt{if} es de la evaluación de la primera expresión RCFAEL recibida. En caso de que sea \texttt{(boolV \#f)}, la evaluación del \texttt{if} es la evaluación de la segunda expresión RCFAEL recibida. \\

\item \grade{1} \textbf{equal?} Agrega al parser y al intérprete la variante \texttt{equal?}, que recibe dos expresiones RCFAEL. Si las dos expresiones son \texttt{numV}, compara si son los mismos números regresando \texttt{(boolV \#t)} o \texttt{(boolV \#f)} según sea el caso. Si son expresiones booleanas, sólo si son iguales valores \texttt{boolV}. Al implementar las listas, \texttt{equal?} aplicado a dos listas debe comparar elemento por elemento si son iguales, en caso de que no lo sean o sean de tamaños distintos, regresar \texttt{(boolV \#f)}, de lo contrario \texttt{(boolV \#t)}. Cualquier otro caso de uso de \texttt{isequal?} no es válido y debes regresar un mensaje de error que diga ``La aplicación de equal? no es adecuada''. En tu README explica qué inconvenientes tendría implementar \textbf{equal?} como parte de la variante de tipo \texttt{binop}. \\

\item \grade{1} \textbf{op, binop} Terminar de implementar todas las operaciones unarias y binarias indicadas en la definición de la sintaxis de RCFAEL. \\

\item \grade{3} \textbf{RCFAEL} Implementar los ambientes recursivos por medio de cajas para la variante de tipo \emph{rec} \\

\item \grade{2} \textbf{Procedurales} Implementar los ambientes recursivos por medio de \emph{procedimientos} para la variante de tipo \emph{rec} \\

\end{enumerate}

\end{document}
