\documentclass{article}
\usepackage[left=2cm,right=2cm,top=3cm,bottom=3cm,letterpaper]{geometry}
\usepackage[spanish]{babel}
\usepackage[utf8]{inputenc}

\usepackage{verbatim, array}
\usepackage{hyperref}
\usepackage{amsmath, amsfonts, amssymb}
\usepackage{graphicx}
\usepackage[T1]{fontenc}

\newcommand{\gradeone}{(\textbf{1pt}) }
\newcommand{\grade}[1]{(\textbf{#1pts}) }
\newcommand{\jimage}[2]{\includegraphics[width=#1\textwidth]{#2}\vskip10pt}
\newcommand{\jcimage}[2]{\begin{center}\includegraphics[width=#1\textwidth]{#2}\end{center}\vskip10pt}

\author{Profesora: Karla Ramírez Pulido \and
  Ayudante teoría: Joshua Emmanuel Mendoza Mendieta \and
  Ayudante laboratorio: Héctor Enrique Gómez Morales}
\title{Practica 6 - Programacion Orientada a Objetos}
\date{Fecha de inicio: 22 de mayo de 2015\\
  \textbf{Fecha de entrega: 12 de junio de 2015}}
\begin{document}
\maketitle
\section{Instrucciones}

Una grafica es un par ordenado $G = (V, E)$ compuesto de un conjunto $V$ de vertices y un conjunto $E$ de aristas.
En esta practica la grafica se puede representar en tres tipos de formatos.


\textbf{Formato CSV:} En el primer renglon se indica si la grafica es dirigida (direct=1) o no (direct=0).
En cada renglon subsecuente se define una arista de la grafica, las dos primeras columnas indican
el vertice origen y el vertice destino de la arista. La tercera columna indica el peso de la arista.

\begin{verbatim}
direct=0
"a", "b", 11
"a", "e", 1
"a", "f", 4
"b", "c", 8
"b", "g", 8
"c", "d", 3
"c", "h", 3
"d", "e", 3 
"d", "i", 2
"e", "j", 1
"f", "h", 9
"f", "i", 7
"g", "i", 4
"g", "j", 1
"h", "j", 9
\end{verbatim}


\textbf{Formato JSON:} Es el formato mas popular para el intercambio de informacion en Web. se tienen tres llaves: direct, vertices y edges. La primera direct indica con un entero si la grafica es dirigida (direct: 1) o si es no dirigida (direct: 0). En la llave vertices se tiene un arreglo con los todos los vertices de la grafica, finalmente en la
llave edges se tiene un arreglo con las aristas de la grafica. 

\begin{verbatim}
{
  "direct": 0
  "vertices": ["a", "b", "c", "d", "e", "f", "g", "h", "i", "j"]
  "edges": [
    ["a", "b", 11],
    ["a", "e", 1],
    ["a", "f", 4],
    ["b", "c", 8],
    ["b", "g", 8],
    ["c", "d", 3],
    ["c", "h", 3],
    ["d", "e", 3],
    ["d", "i", 2],
    ["e", "j", 1],
    ["f", "h", 9],
    ["f", "i", 7],
    ["g", "i", 4],
    ["g", "j", 1],
    ["h", "j", 9]
  ]
}
\end{verbatim}

\textbf{Formato xml:}
\begin{verbatim}
<?xml version="1.0" encoding="UTF-8"?>
<!DOCTYPE graph PUBLIC "-//FC//DTD matrix//EN" "./graph.dtd">
<graph direct="0">
  <vertex label="a"/>
  <vertex label="b"/>	
  <vertex label="c"/>
  <vertex label="d"/>
  <vertex label="e"/>
  <vertex label="f"/>
  <vertex label="g"/>
  <vertex label="h"/>
  <vertex label="i"/>
  <vertex label="j"/>
  <edge source="a" target="b" weight="11"/>
  <edge source="a" target="e" weight="1"/>
  <edge source="a" target="f" weight="4"/>
  <edge source="b" target="c" weight="8"/>
  <edge source="b" target="g" weight="8"/>
  <edge source="c" target="d" weight="3"/>
  <edge source="c" target="h" weight="3"/>
  <edge source="d" target="e" weight="3"/>
  <edge source="d" target="i" weight="2"/>
  <edge source="e" target="j" weight="1"/>
  <edge source="f" target="h" weight="9"/>
  <edge source="f" target="i" weight="7"/>
  <edge source="g" target="i" weight="4"/>
  <edge source="g" target="j" weight="1"/>
  <edge source="h" target="j" weight="9"/>
</graph>
\end{verbatim}

En esta practica sera la implementacion de clases para la representacion de graficas (dirigidas y no dirigidas), la implementacion de funcionalidad basica del algoritmo de Kruskal para obtener el arbol de peso minimo de la grafica. 

Esta práctica se puede entregar en equipos de a lo más tres personas.

Esta práctica debe ser implementada con cualquier lenguaje orientado a objetos excepto \textbf{Java}.

Se debe incluir un archivo README que indique las instrucciones para correr su programa.

\section{Ejercicios}

\begin{enumerate}

\item \grade{3} \textbf{Graph} Implementar una clase que represente una grafica debe tener por lo menos las siguientes funciones:

  \begin{itemize}
  \item \textbf{directed} Una funcion que regresa u booleano, true si la grafica es dirigida y false si no lo es.
  \item \textbf{vertices} Una funcion que regresa un arreglo con todos los vertices de la grafica.
  \item \textbf{edges} Una funcion que regresa todas las aristas de la grafica.
  \end{itemize}

\item \grade{1.5} \textbf{Vertex} Clase que representa un vertice de la grafica, debe tener por lo menos las siguientes funciones:

  \begin{itemize}
  \item \textbf{neighbours} Una funcion que regresa los vertices adyacentes del vertice dado.
  \item \textbf{grade} Una funcion que indica el grado del vertice.
  \end{itemize}

\item \grade{1.5} \textbf{Edge} Clase que representa una arista de la grafica, debe tener por lo menos las siguientes funciones:

  \begin{itemize}
  \item \textbf{svertex} Regresa el vertice origen de la arista.
  \item \textbf{tvertex} Regresa el vertice destino de la arista.
  \item \textbf{weight} El peso de la arista
  \end{itemize}

 \item \grade{3} \textbf{GraphReader} Deben implementar una clase que tome una ruta a un archivo (ya sea XML, JSOn o CSV) y que regrese un objeto de la clase \texttt{Graph}

 \item \grade{1} \textbf{Kruskal} Se debe de implementar una funcion que tome una grafica y nos regrese el arbol de peso minimo.

   Finalmente al correr su programa se debe de cargar por cada formato la grafica de Peteren y imprimir lo siguiente:
   \begin{itemize}
   \item Los vertices de la grafica
   \item Las aristas de la grafia con sus pesos
   \item Las aristas del arbol de peso minimo de la grafica, con su peso total.
   \end{itemize}

 \textbf{IMPORTANTE:} Se restaran 2 puntos de no presentarse polimorfismo en alguno de los puntos anteriormente mencionados.

\end{enumerate}

\end{document}
