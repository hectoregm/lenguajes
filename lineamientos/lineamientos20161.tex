\documentclass[10pt]{article}
\usepackage{fancyhdr}
\usepackage[utf8]{inputenc}
\usepackage[spanish]{babel}
\usepackage{anysize}
\usepackage{graphicx}
\usepackage{hyperref}
\usepackage[export]{adjustbox}
\marginsize{2cm}{2cm}{2cm}{2cm}
\pagestyle{fancy}

%\sloppy
\lhead{Lineamientos} 
\chead{} 
\rhead{Lenguajes de Programación} 
\lfoot{}
\cfoot{\thepage} 
\rfoot{} 
%\setlength{\headrulewidth}{0.4pt}
%\setlength{\footrulewidth}{0.4pt}
\usepackage{array}
\newcolumntype{C}[1]{>{\centering\let\newline\\\arraybackslash\hspace{0pt}}b{#1}}
\begin{document}
\begin{center}
\begin{tabular}[h]{c C{10cm} c}
\includegraphics[scale=.2]{cover.png} & 
\textbf{{\Large Lenguajes de \newline Programación}} 
\newline \textbf{Profesor}: Karla Ramírez Pulido 
\newline \textbf{Ayudante de Teoría}: Héctor Enrique Gómez Morales
\newline \textbf{Ayudante le Laboratorio}: Héctor Enrique Gómez Morales
\newline Inicio de semestre: 10 de agosto de 2015
\newline Fin de semestre: 27 de noviembre de 2015 &
\includegraphics[scale=.4]{cover2.png}
\end{tabular}
\end{center}

\begin{itemize}
\item \textbf{Acerca del Curso}
  
  \begin{itemize}
  \item  Nombre de la materia: Lenguajes de Programación.
  \item Tipo de materia: Obligatoria de 5to. semestre en Ciencias de
    la Computación.
  \item Clave: 1536.
  \item Grupo: 7050.
  \item Salón: P-211.
  \item Laboratorio: Laboratorio de Ciencias de la Computación.
  \item Horario de Clase: Lunes, Martes y Viernes de 11:00 a 12:00 hrs.
  \item Horario de Ayudantía: Miércoles y Jueves de 11:00 a 12:00 hrs.
  \item Horario de Laboratorio: Miércoles de 16:00 a 18:00 hrs.
  \end{itemize}

\item \textbf{Lenguaje de Programación}

El principal lenguaje de programación que se usará será Racket (un
descendiente de Scheme), aunque habrán prácticas que se podrán realizar en el lenguaje
de programación elegido por el equipo. Para las prácticas en Racket se utilizará el
intérprete del PLT (también conocido como DrRacket) versión 6.2. El dialecto que se
usará en el curso (a menos que se especifique uno diferente para alguna práctica)
será PLAI.

\item \textbf{Evaluación - Teoría}
  
  La evaluación de la parte teórica se dividirá en dos secciones:

  \begin{enumerate}
  \item Cada 15 días los miércoles al inicio de la clase se entregará un reporte de lectura sobre algún texto o recurso literiario especificado en clase. Estos textos complementarán el curso como material de apoyo y serán evaluados en función del contenido y sus referencias bibliográficas. El objetivo es plasmar de manera simple los conceptos tratados.

    La rúbrica de evaluación para dicho reporte es la siguientes:

    \begin{itemize}
    \item 7 pts Presentación escrita clara del tema, estructura definida del texto y coherencia en el flujo de ideas.
    \item 3 pts Texto plenamente fundamentado en referencias bibliográficas. 
    \end{itemize}

  \item Los días miércoles que se entrega la tarea, 20 minutos antes de terminar la clase se proyectará en cañón una breve evaluación del tema tratado, contando con preguntas de opción múltiple y una o dos preguntas abiertas por contestar.

  \end{enumerate}
\pagebreak

\item \textbf{Forma de Evaluación}
  
  La forma de evaluar a cada alumno está dada por los rubros:
  exámenes, prácticas y teoría.
  
  \begin{itemize}
  \item Exámenes: 40\%
  \item Prácticas: 30\%
  \item Tareas: 30\%
  \end{itemize}
    
  Las siguientes son restricciones generales sobre proceso de calificación y entrega de prácticas, reportes, evaluaciones y exámenes:
  
  \begin{itemize} 
  \item No habrá reposición de exámenes, reportes, evaluaciones, ni prácticas.
  \item No habrá examen final.
  \item La fecha y forma de entrega de las prácticas y parte teórica será \textbf{estricta e inapelable}, esto es, no habrá prórrogas.
  \item Los exámenes serán evaluados individualmente; las prácticas, los reportes escritos y las evaluaciones presenciales serán por equipos de 2 personas mínimo a 3 personas máximo.
  \item Cualquier reporte que incluya traducciones automáticas (o carentes de sentido) tendrá calificación de 0 automáticamente.
  \item Para tener derecho a presentar cada examen en el curso se deberán haber entregado todas y cada uno de los reportes previos al mismo.
   \end{itemize}
    
  Las siguientes reglas  no se pueden cambiar bajo ninguna circunstancia. Incurrir en el incumplimiento de alguna de ellas implica obtener automáticamente una calificación final de 5:
  \begin{itemize}
  \item Está estrictamente prohibido cometer plagio. El plagio implica reproducir (o traducir de manera literal) íntegra o parcialmente tareas, prácticas, publicaciones, anotaciones, etc. sin señalar de forma precisa a su autor original e indicar el documento o lugar en el que se encontró. 
  \item Es necesario cumplir al menos con el 80\% de asistencia incluyendo las ayudantías.
  \end{itemize} 
  
  Las calificaciones finales serán definitivas. El único caso en el que se podrá obtener NP será el no haber entregado ninguna tarea, práctica o examen.
  
  \item \textbf{Lista de Correo}

  La lista de correo del curso es: \textbf{lenguajes@googlegroups.com} donde se
  podrán discutir las dudas de prácticas y temas relacionados
  con el curso entre otras cosas.

\item \textbf{Página del Curso}

  Por medio de la página del curso será posible ver información relacionada
  con el mismo, de tal suerte que no habrá complicación para descargar las
  prácticas y tareas que se hayan encargado durante el semestre. A su
  vez, se publicará información relacionada con calificaciones,
  lineamientos, el uso del taller y noticias mencionadas en
  clase. La página estará disponible en: \\
  \url{http://valhalla.fciencias.unam.mx/lenguajes} 

\end{itemize}
\end{document}
