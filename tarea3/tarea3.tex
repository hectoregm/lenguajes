\documentclass{article}
\usepackage[left=2cm,right=2cm,top=3cm,bottom=3cm,letterpaper]{geometry}
\usepackage[spanish]{babel}
\usepackage[utf8]{inputenc}
\author{Profesora: Karla Ramírez Pulido \and
  Ayudante: Héctor Enrique Gómez Morales}
\title{Tarea 2}
\date{Fecha de inicio: 4 de noviembre de 2015\\
  \textbf{Fecha de entrega: 18 de noviembre de 2015}}
\begin{document}
\maketitle
\section*{Problema I}
Considera el siguiente programa:

\begin{verbatim}
(+ 1 (first (cons true empty)))
\end{verbatim}

Este programa tiene un error de tipos.

Genera restricciones para este programa. Aísla el conjunto mas pequeño de
estas restricciones tal que, resultas juntas, identifiquen el error de tipos.

\section*{Problema II}
Considera la siguiente expresión con tipos:

\begin{verbatim}
{fun {f : C1 } : C2
  {fun {x : C3 } : C4
    {fun {y : C5 } : C6
      {cons x {f {f y}}}}}}
\end{verbatim}

Dejamos los tipos sin especificar (Cn) para que sean llenados por el proceso
de inferencia de tipos. Deriva restricciones de tipos para el programa anterior.
Luego resuelve estas restricciones. A partir de estas soluciones, rellena los
valores de las Cn. Asegurate de mostrar todos los pasos especificados por los
algoritmos (i.e., escribir la respuesta basandose en la intuicion o el conocimiento
es insuficiente). Deberas usar variables de tipo cuando sea necesario.
Para no escribir tanto, puedes etiquetar cada expresion con una variable de tipos
apropiada, y presentar el resto del algoritmo en terminos solamente de estas
variables de tipos.

\section*{Problema III}
Considera los juicios de tipos discutidos en clase para un lenguaje gloton
(en el capitulo de Juicios de Tipos del libro de Shriram). Considera ahora la
version perezosa del lenguaje. Pon especial atencion a las reglas de tipado para:

\begin{itemize}
\item definicion de funciones
\item aplicacion de funciones
\end{itemize}

Para cada una de estas, si crees que la regla original no cambia, explica por que no
(Si crees que ninguna de las dos cambia, puedes responder las dos partes juntas).
Si crees que algun otro juicio de tipos debe cambiar, mencionalo tambien.
\end{document}
