\documentclass{article}
\usepackage[left=2cm,right=2cm,top=3cm,bottom=3cm,letterpaper]{geometry}
\usepackage[spanish]{babel}
\usepackage[utf8]{inputenc}
\author{Profesora: Karla Ramírez Pulido \and
  Ayudante: Héctor Enrique Gómez Morales}
\title{Practica 1 - Principios de Racket y Recursión}
\date{Fecha de inicio: 19 de agosto de 2015\\
  \textbf{Fecha de entrega: 2 de septiembre de 2015}}
\begin{document}
\maketitle
\section{Instrucciones}
En esta práctica se tienen once ejercicios.

Esta práctica debe ser implementada con la variante plai, es decir
su archivo con terminación \textit{.rkt} debe tener como primer linea lo siguiente:
\texttt{\#lang plai}

Todos los ejercicios requieren contar con pruebas mediante el uso de
la función \texttt{test}:
\begin{verbatim}
(test <result-expr> <expected-expr>)
\end{verbatim}

En donde \textit{result-expr} es una expresión que se evalúa y su valor obtenido
es comparado con valor obtenido de la expresión \textit{expected-expr} que es el
valor esperado de la prueba. Si las dos expresiones evalúan a lo mismo la prueba
imprime el éxito de la prueba, en caso contrario indicar un error.

\begin{verbatim}
> (test (+ 1 2) 3)
(good (+ 1 2) 3 3 "at line 34")

> (test (+ 1 2) 4)
(bad (+ 1 2) 3 4 "at line 36")
\end{verbatim}

Cada ejercicio debe contar al menos con \textbf{cinco} pruebas.

\section{Ejercicios}
\textbf{Sección I.} Define las funciones que se te piden. Puedes crear y utilizar
funciones auxiliares. No puedes utilizar funciones de Racket que
resuelvan \textbf{directamente} los ejercicios.
\begin{itemize}

\item $\textbf{pow}$ - Define la función pow tal que toma dos números enteros
  positivos  $z$ y $w$ y regresa el numero que se obtiene de elevar el numero $z$ a la potencia $w$, \textit{i.e.}
\begin{verbatim}
> (pow 2000 0)
1
> (pow  2 3)
8
> (pow 8 6)
262144
\end{verbatim}
\newpage

\item $\textbf{average}$ - Dado una lista no vacía de números, regresar el
  promedio de esta, \textit{i.e.}
\begin{verbatim}
> (average ’(5))
5
> (average ’(3 2 6 2 1 7 2 1))
3
> (average ’(10 7 13))
10
\end{verbatim}

\item $\textbf{primes}$ - Dado un numero entero positivo regresar una lista los
números primos contenidos entre 2 y el numero entero dado, \textit{i.e.}
\begin{verbatim}
> (primes 30)
’(2 3 5 7 11 13 17 19 23 29)
> (primes 11)
'(2 3 5 7 11)
> (primes 1)
'()
\end{verbatim}

\item $\textbf{zip}$ - Dadas dos listas, regresar una lista cuyos elementos son
  listas de tamaños dos, tal que par la i-ésima lista, el primer
  elemento es el i-ésimo de la primera lista original y el segundo
  elemento es el i-ésimo de la segunda lista original, si una lista es
de menor tamaño que la otra, la lista resultante es del tamaño de la
menor, y si una de las listas es vacía, regresar una lista vacía,
\textit{i.e.}
\begin{verbatim}
> (zip ’(1 2) ’(3 4))
’((1 3) (2 4))
> (zip ’(1 2 3) ’())
’()
> (zip ’() ’(4 5 6))
’()
> (zip ’(8 9) ’(3 2 1 4))
’((8 3) (9 2))
> (zip ’(8 9 1 2) ’(3 4))
’((8 3) (9 4))
\end{verbatim}

\item $\textbf{reduce}$ - Dada una función de aridad 2 y una lista de
  n elementos, regresar la evaluación de la función encadenada de
  todos los elementos, \textit{i.e.}
\begin{verbatim}
> (reduce + ’(1 2 3 4 5 6 7 8 9 10))
55
> (reduce zip ’((1 2 3) (4 5 6) (7 8 9)))
’((1 (4 7)) (2 (5 8)) (3 (6 9)))
\end{verbatim}

\end{itemize}

\textbf{Sección II.} Define las funciones que se te piden. Puedes crear
y utilizar funciones auxiliares. Para el manejo de listas sólo puedes
utilizar funciones básicas de Racket como son \texttt{car, cdr, cons,
  list, empty, empty?, if, let, let*}

\begin{itemize}
\item $\textbf{mconcat}$ - Dadas dos listas, regresa la concatenación
  de la primera con la segunda, \textit{i.e.}
\begin{verbatim}
> (mconcat ’(1 2 3) ’(4 5 6))
’(1 2 3 4 5 6)
> (mconcat ’() ’(4 5 6))
’(4 5 6)
> (mconcat ’(1 2 3) ’())
’(1 2 3)
\end{verbatim}

\item $\textbf{mmap}$ - Dada una función de aridad 1 y una lista,
  regresar una lista con la aplicación de la función a cada uno de los
  elementos de la lista original, \textit{i.e.}
\begin{verbatim}
> (mmap add1 ’(1 2 3 4))
’(2 3 4 5)
> (mmap car ’((1 2 3) (4 5 6) (7 8 9)))
’(1 4 7)
> (mmap cdr ’((1 2 3) (4 5 6) (7 8 9)))
’((2 3) (5 6) (8 9))
\end{verbatim}

\item $\textbf{mfilter}$ - Dado un predicado de un argumento y una
  lista, regresar la lista original sin los elementos que al aplicar
  el predicado, regresa falso (un predicado es una función que
  regresa un valor booleano), \textit{i.e.}
\begin{verbatim}
> (mfilter (lambda (x) (not (zero? x))) ’(2 0 1 4 0))
’(2 1 4)
> (mfilter (lambda (l) (not (empty? l))) ’((1 4 2) () (2 4) ()))
’((1 4 2) (2 4))
> (mfilter (lambda (n) (= (modulo n 2) 0)) ’(1 2 3 4 5 6))
’(2 4 6)
\end{verbatim}

\item $\textbf{any?}$ - Dado un predicado de un argumento y
  una lista se debe regresar $\#t$ cuando por lo menos un
  elemento de la lista regresa $\#t$ con el predicado dado.
  En caso contrario regresa $\#f$, \textit{i.e.}
\begin{verbatim}
> (any? number? '())
#f
> (any? number? '(a b c d 1))
#t
> (any? symbol? '(1 2 3 4))
#f
\end{verbatim}

\item $\textbf{every?}$ - Dado un predicado de un argumento y
  una lista se debe regresar solamente $\#t$ cuando cada uno
  de los elementos de la lista regresa $\#t$ para el predicado dado.
  En caso contrario regresa $\#f$, \textit{i.e.}
\begin{verbatim}
> (every? number? '())
#t
> (every? number? '(1 2 3))
#t
> (every? number? '(1 2 3 a))
#f
> (every? symbol? '(1 2 3 a))
#f
\end{verbatim}

\item $\textbf{mpowerset}$ - Dada una lista, regresar otra como el
  conjunto potencia de la original (un conjunto potencia de S es un
  conjunto de subconjuntos de S), \textit{i.e.}
\begin{verbatim}
> (mpowerset ’())
’(())
> (mpowerset ’(1))
’(() (1))
> (mpowerset ’(1 2))
’(() (1) (2) (1 2))
> (mpowerset ’(1 2 3))
’(() (1) (2) (3) (1 2) (1 3) (2 3) (1 2 3))
\end{verbatim}
\end{itemize}
\end{document}
