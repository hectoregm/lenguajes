\documentclass{article}
\usepackage[left=3cm,right=3cm,top=3cm,bottom=3cm,letterpaper]{geometry}
\usepackage[spanish]{babel}
\usepackage[utf8]{inputenc}
\author{Profesora: Karla Ramírez Pulido \and
  Ayudante: Héctor Enrique Gómez Morales}
\title{Lineamientos Tareas y Laboratorio  Lenguajes 2016-1}
\begin{document}
\maketitle

\section{Puntos Generales}

\begin{itemize}
\item Las prácticas y tareas serán por equipos de 2 personas mínimo a 3 personas máximo.
\item Cada equipo tiene que tener un \textbf{único} repositorio en Github, el nombre del repositorio debe tener el siguiente formato:
\begin{verbatim}
lenguajes20161_<UUID>
\end{verbatim}
Donde donde \textbf{UUID} es cualquier cadena para diferenciar su repositorio.

\item En la raíz del repositorio debe haber un archivo \textbf{README}
  que contenga la lista de los integrantes del equipo con los siguientes datos:
  nombre completo, correo electrónico y numero de cuenta.

\item Aparte de agregar a cada unos de sus integrantes como colaboradores al repositorio, me tienen que agregar como colaborador de su repositorio (\textbf{@hectoregm}).
\item Cualquier omisión al formato de entrega de las practicas y tareas será
sancionado.
\end{itemize}

\section{Prácticas}

En la raíz del repositorio debe haber un directorio \textbf{Practicas}
en la cual deben estarán sus practicas del curso.

Cada practica debe estar contenida en su directorio del estilo \texttt{practica1, practica2, ... practicaN} donde cada directorio contendrá los archivos correspondientes a cada práctica, los archivos de Racket deben llevar la extensión \texttt{.rkt}.

\section{Tareas}

En la raíz del repositorio debe haber un directorio \textbf{Tareas}
en la cual estarán sus tareas del curso

Si la tarea es en equipo el nombre del archivo debe tener el siguiente formato \texttt{TareaN.pdf}. Si la tarea es individual entonces cada uno debe subir un pdf con el siguiente formato:
\begin{verbatim}
TareaN_<Apellido Paterno><Apellido Materno>_<Nombres>.pdf
\end{verbatim}

\section{Contacto}

Cualquier duda mandar un correo a \textbf{hectoregm@gmail.com}, todo correo debe tener el siguiente sujeto o asunto: \texttt{[Lenguajes] Asunto X} esto para facilitar la mas rápida respuesta a sus dudas.
\end{document}
