\documentclass{article}
\usepackage[left=2cm,right=2cm,top=3cm,bottom=3cm,letterpaper]{geometry}
\usepackage[spanish]{babel}
\usepackage[utf8]{inputenc}
\author{Profesora: Karla Ramírez Pulido\\
  Ayudante teoría: Joshua Emmanuel Mendoza Mendieta\\
  Ayudante laboratorio: Héctor Enrique Gómez Morales}
\title{Practica 3 - Data Types Parte 2 de 2}
\date{Fecha de inicio: 13 de febrero de 2015\\
  \textbf{Fecha de entrega: 27 de febrero de 2015}}
\begin{document}
\maketitle
\section{Instrucciones}
En esta práctica se tienen doce ejercicios, los primeros once son
obligatorios siendo solamente el ultimo opcional con valor de un
punto extra. Por lo tanto la calificación máxima que se puede obtener en
esta práctica es 11.

Esta practica se puede entregar en equipos de a lo máximo dos
personas, pero se recomienda que esta practica la hagan de forma
individual.

Esta práctica debe ser implementada con la variante plai, es decir
su archivo rkt debe tener como primer linea lo siguiente:
\texttt{\#lang plai}. Pueden utilizar sólo la paquetería básica de
\texttt{racket/base}, las funciones que se implementaron en la
practica 1 y funciones auxiliares que ustedes definan.

Todos los ejercicios requieren contar con pruebas mediante el uso de
la función \texttt{test}:
\begin{verbatim}
(test <result-expr> <expected-expr>)
\end{verbatim}

En donde \textit{result-expr} es la expresión que se evalúa a cierto
valor que es comparado con \textit{expected-expr} que es otra
expresión que evalúa al valor esperado. Si las dos expresiones evalúan
a lo mismo la prueba imprime el éxito de la prueba, en caso contrario
indicar un error.

\begin{verbatim}
> (test (+ 1 2) 3)
(good (+ 1 2) 3 3 "at line 34")

> (test (+ 1 2) 4)
(bad (+ 1 2) 3 4 "at line 36")
\end{verbatim}

Cada ejercicio debe contar al menos con cinco pruebas.

\section{Ejercicios}
\textbf{Sección I. Tipos de datos recursivos y no recursivos} Define
los siguientes tipos de datos que se te piden.

\begin{itemize}
\item $\textbf{Array}$ - Definir un tipo de dato \texttt{Array}
  que tenga un constructor de tipo \texttt{MArray}. El entero sirve
  para definir el tamaño del arreglo., \textit{i.e.}
\begin{verbatim}
> (MArray 4 '(1 2 3))
(Marray 4 '(1 2 3))
\end{verbatim}

\newpage

\item $\textbf{List}$ - Definir un tipo de dato recursivo llamado
  \texttt{MList} que tenga a la lista vacía \texttt{MEmpty} y el
  constructor de tipo \texttt{MCons}, \textit{i.e.}
\begin{verbatim}
> (MEmpty)
(MEmpty)
> (MCons 1 (MCons 2 (MCons 3 (MEmpty))))
(MCons 1 (MCons 2 (MCons 3 (MEmpty))))
> (MCons 7 (MCons 4 (MCons 10 (MEmpty))))
(MCons 7 (MCons 4 (MCons 10 (MEmpty))))
\end{verbatim}

\item $\textbf{NTree}$ - Definir un tipo de dato recursivo llamado
  \texttt{NTree} que tenga como una hoja nula \texttt{TLEmpty} y un
  constructor de tipo \texttt{NodeN} (estos árboles son n-ários), \textit{i.e.}
\begin{verbatim}
> (TLEmpty)
(TLEmpty)
> (NodeN 1 (list (TLEmpty) (TLEmpty) (TLEmpty)))
(NodeN 1 (list (TLEmpty) (TLEmpty) (TLEmpty)))
> (NodeN 1 (list (NodeN 2 (list (TLEmpty))) 
                 (NodeN 3 (list (TLEmpty))) 
                 (NodeN 4 (list (TLEmpty) (TLEmpty) (TLEmpty)))))
(NodeN
 1
 (list
  (NodeN 2 (list (TLEmpty)))
  (NodeN 3 (list (TLEmpty)))
  (NodeN 4 (list (TLEmpty) (TLEmpty) (TLEmpty)))))
\end{verbatim}

\item $\textbf{Position}$ - Define un tipo de dato llamado
  \texttt{Position} que tenga un constructor de tipo \texttt{2D-Point}
  que toma dos números reales que indican una posición
  en el plano cartesiano, \textit{i.e.}
\begin{verbatim}
> (2D-Point 0 0)
(2D-Point 0 0)
> (2D-Point 1 (sqrt 2))
(2D-Point 1 1.4142135623730951)
\end{verbatim}

\item $\textbf{Figure}$ - Define un tipo de dato llamado
  \texttt{Figure}
  que tenga tres constructores:
  \begin{itemize}
    \item{\texttt{Circle} - Un constructor que toma un centro dada por un
      \texttt{Position} y un radio.}
    \item{\texttt{Square} - Un constructor que toma una posición de la
      esquina superior izquierda del cuadrado y una longitud.}
    \item{\texttt{Rectangle} - Un constructor que toma una posición de
    la esquina superior izquierda del rectángulo, su ancho y su largo.}
  \end{itemize}
\begin{verbatim}
> (Circle (2D-Point 2 2) 2)
(Circle (2D-Point 2 2) 2)
> (Square (2D-Point 0 3) 3)
(Square (2D-Point 0 3) 3)
> (Rectangle (2D-Point 0 2) 2 3)
(Rectangle (2D-Point 0 2) 2 3)
\end{verbatim}


\textbf{Sección II. Funciones sobre Datos} Define
las siguientes funciones que se te piden. Puedes crear y utilizar
funciones auxiliares. No puedes utilizar funciones de Racket que
resuelvan \textbf{directamente} los ejercicios.

\item $\textbf{setvalueA}$ - Dado un arreglo de tipo \texttt{Array},
  una posición y un valor numérico v, regresar otro arreglo con el
  valor v intercambiado en la posición indicada del arreglo original.
  En caso de que la posición sea igual o mayor al tamaño especificado
  en el arreglo, regresar un error \texttt{Out of bounds},
  \textit{i.e.}
\begin{verbatim}
> (define ar (MArray 5 '(0 0 0 0 0)))
> ar
(MArray 5 '(0 0 0 0 0))
> (setvalueA ar 2 4)
(MArray 5 '(0 0 4 0 0))
> (setvalueA ar 4 4)
(MArray 5 ’(0 0 0 0 4))
> (setvalueA ar 5 4)
. . setvalueA: Out of bounds
\end{verbatim}

\item $\textbf{printML}$ - Dada una lista de tipo \texttt{MList},
  imprimirla en un formato legible. Puedes utilizar las funciones para
  manipular strings y la función \texttt{$\sim a$}, \textit{i.e.}
\begin{verbatim}
> (printML (MEmpty))
"[]"
> (printML (MCons 7 (MEmpty)))
"[7]"
> (printML (MCons 7 (MCons 4 (MEmpty))))
"[7, 4]"
> (printML (MCons (MCons 1 (MCons 2 (MEmpty))) (MCons 3 (MEmpty))))
"[[1, 2], 3]"
\end{verbatim}

\item $\textbf{concatML}$ - Dadas dos listas de tipo \texttt{MList},
  regresar la concatenación, \textit{i.e.}
\begin{verbatim}
> (concatML (MCons 7 (MCons 4 (MEmpty))) (MCons 1 (MEmpty)))
(MCons 7 (MCons 4 (MCons 1 (MEmpty))))
> (concatML (MCons 7 (MCons 4 (MEmpty))) (MCons 1 (MCons 10 (MEmpty))))
(MCons 7 (MCons 4 (MCons 1 (MCons 10 (MEmpty)))))
\end{verbatim}

\item $\textbf{lengthML}$ - Dada una lista de tipo MLista, regresar la
  cantidad de elementos que tiene, \textit{i.e.}
\begin{verbatim}
> (lengthML (MEmpty))
0
> (lengthML (MCons 7 (MCons 4 (MEmpty))))
2
\end{verbatim}

\item $\textbf{area}$ - Dada una figura del tipo \texttt{Figure},
  regresar el área de la figura, \textit{i.e.}
\begin{verbatim}
> (area (Circle (2D-Point 5 5) 4))
50.26548245743669
> (area (Square (2D-Point 0 0) 20))
400
> (area (Rectangle (2D-Point 3 4) 5 10))
50
\end{verbatim}

\item $\textbf{in-figure?}$ - Dada una figura \texttt{fig} del tipo \texttt{Figure}
  y una posición \texttt{p} del tipo \texttt{2D-Point} regresa \#t si \texttt{p} esta
  dentro de \texttt{fig} y \#f en caso contrario, \textit{i.e.}
\begin{verbatim}
> (in-figure? (Square (2D-Point 5 5) 4) (2D-Point 6 6))
#t
> (in-figure? (Rectangle (2D-Point 5 5) 4 6) (2D-Point 4 4))
#f
\end{verbatim}

\end{itemize}

\textbf{Punto extra:} \texttt{in-figure-3D?} - Dados los siguientes tipos de datos:
\begin{itemize}
\item{ \texttt{3D-Point} que define un punto en 3D.}
\item{\texttt{Figure3D} que define una figura en 3D que tenga al menos tres variantes (por ejemplo: Sphere, Cube)}.
\end{itemize}
Entonces se debe implementar \texttt{in-figure-3D?} que tome una figura \texttt{fig} del tipo \texttt{Figure3D} y una posición \texttt{p} del tipo \texttt{3D-Point} entonces regresa \#t si \texttt{p} esta dentro de \texttt{fig} y \#f en caso contrario
\end{document}
