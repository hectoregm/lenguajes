\documentclass{article}
\usepackage[left=2cm,right=2cm,top=3cm,bottom=3cm,letterpaper]{geometry}
\usepackage[spanish]{babel}
\usepackage[utf8]{inputenc}

\usepackage{verbatim, array}
\usepackage{hyperref}
\usepackage{amsmath, amsfonts, amssymb}
\usepackage{graphicx}
\usepackage[T1]{fontenc}

\newcommand{\jimage}[2]{\includegraphics[width=#1\textwidth]{#2}\vskip10pt}
\newcommand{\jcimage}[2]{\begin{center}\includegraphics[width=#1\textwidth]{#2}\end{center}\vskip10pt}

\author{Profesora: Karla Ramírez Pulido \and
  Ayudante: Héctor Enrique Gómez Morales}
\title{Practica 3 - Tipos de Datos Parte 2 de 2}
\date{Fecha de inicio: 2 de septiembre de 2015\\
  \textbf{Fecha de entrega: 17 de septiembre de 2015}}
\begin{document}
\maketitle
\section{Instrucciones}
En esta práctica se tienen trece ejercicios.

Esta práctica debe ser implementada con la variante plai, es decir
su archivo con terminación \textit{.rkt} debe tener como primer linea lo siguiente:
\texttt{\#lang plai}. Pueden utilizar sólo la paquetería básica de
\texttt{racket/base} y funciones auxiliares que ustedes definan.

Todos los ejercicios requieren contar con pruebas mediante el uso de
la función \texttt{test}:
\begin{verbatim}
(test <result-expr> <expected-expr>)
\end{verbatim}

En donde \textit{result-expr} es una expresión que se evalúa y su valor obtenido
es comparado con valor obtenido de la expresión \textit{expected-expr} que es el
valor esperado de la prueba. Si las dos expresiones evalúan a lo mismo la prueba
imprime el éxito de la prueba, en caso contrario indicar un error.

\begin{verbatim}
> (test (+ 1 2) 3)
(good (+ 1 2) 3 3 "at line 34")

> (test (+ 1 2) 4)
(bad (+ 1 2) 3 4 "at line 36")
\end{verbatim}

Cada ejercicio debe contar al menos con \textbf{cinco} pruebas.

\section{Ejercicios}
\textbf{Sección I. Heart Rate Zones} Para las siguientes funciones, se utilizaran los tipos \verb;HRZ; ,\verb;Coordinate; y \verb;Frame; definidos en \verb;practica3-base.rkt;:
\begin{verbatim}
(define-type HRZ
  [resting (low number?)
           (high number?)]
  [warm-up (low number?)
           (high number?)]
  [fat-burning (low number?)
               (high number?)]
  [aerobic (low number?)
           (high number?)]
  [anaerobic (low number?)
             (high number?)]
  [maximum (low number?)
           (high number?)])

(define-type Coordinate
  [GPS (lat number?)
       (long number?)])

(define-type Frame
  [trackpoint (loc GPS?)
              (hr exact-integer?)
              (zone HRZ?)
              (unix-time exact-integer?)])
\end{verbatim}

El tipo de dato \verb;HRZ; representa las zonas de frecuencia cardiaca,
cada zona tiene como parámetros su mínimo y máximo ritmo cardiaco correspondiente.

El tipo de dato \verb;Coordinate; representa una posición, se tiene el constructor
de tipo \verb;GPS; que representa una posicion GPS que consta de una latitud y una longitud.

El tipo de dato \verb;Frame; representa un instante o marco, se tiene el constructor del tipo \verb;trackpoint; que toma una coordenada GPS, un ritmo cardiaco, una zona de frecuencia cardiaca y una \verb;timestamp; en formato \verb;UNIX time;.

\begin{itemize}
\item $\textbf{zones}$ - Dado el ritmo cardiaco de descanso y el máximo ritmo cardiaco de una persona se debe regresar la lista de zonas de frecuencia cardiaca. La formula utilizada comúnmente es:\\
  $range = max - rest$\\
  $zone\ 0\ (resting) = rest\ (Minimo)$\\
  $zone\ 0\ (resting) = (range * 0.5) - 1\ (Maximo)$\\

  $zone\ (i+1) = rest + range * (0.5 + (0.1 * i))\ (Minimo)$\\
  $zone\ (i+1) = rest + range * (0.5 + (0.1 * (i+1))) - 1\ (Maximo)$,\\
  donde i = $0..4$, \textit{i.e.}

\begin{verbatim}
> (zones 50 180)
(list
 (resting 50 114.0)
 (warm-up 115.0 127.0)
 (fat-burning 128.0 140.0)
 (aerobic 141.0 153.0)
 (anaerobic 154.0 166.0)
 (maximum 167.0 180))
\end{verbatim}

Para los siguientes ejemplos se define:
\begin{verbatim}
> (define my-zones (zones 50 180))
\end{verbatim}

\item $\textbf{get-zone}$ - Dado un símbolo que es el nombre de una zona y una lista de zonas regresar el tipo de dato correspondiente, \textit{i.e.}
\begin{verbatim}
> (get-zone 'anaerobic my-zones)
(anaerobic 154.0 166.0)
> (get-zone 'maximum my-zones)
(maximum 167.0 180)
\end{verbatim}

\newpage

\item $\textbf{bpm->zone}$ - Dado una lista de frecuencias cardiacas y una lista de zonas de frecuencia cardiaca regresar una lista de zonas por cada frecuencia cardiaca en la lista original, \textit{i.e.}
\begin{verbatim}
> (bpm->zone empty my-zones)
'()
> (bpm->zone '(50 60) my-zones) 
(list (resting 50 114.0) (resting 50 114.0))
> (bpm->zone '(140 141) my-zones) 
(list (fat-burning 128.0 140.0) (aerobic 141.0 153.0))
\end{verbatim}

\item $\textbf{create-trackpoints}$ - Dado una lista en la que cada elemento de la lista contiene: un tiempo en formato \verb;UNIX;, una lista con la latitud y longitud y finalmente el ritmo cardiaco. Como segundo parámetro se tiene una lista de zonas cardiacas con lo que se tiene que regresar una lista de \verb;trackpoint;s que contengan la información dada. Se tiene definida la variable \verb;raw-data; con datos reales para sus pruebas, \textit{i.e.}
\begin{verbatim}
> (take raw-data 4))
'((1425619654 (19.4907258 -99.24101) 104)
  (1425619655 (19.4907258 -99.24101) 104)
  (1425619658 (19.4907258 -99.24101) 108)
  (1425619662 (19.4907107 -99.2410833) 106))
> (create-trackpoints (take raw-data 4) my-zones)
(list
 (trackpoint (GPS 19.4907258 -99.24101) 104 (resting 50 114.0) 1425619654)
 (trackpoint (GPS 19.4907258 -99.24101) 104 (resting 50 114.0) 1425619655)
 (trackpoint (GPS 19.4907258 -99.24101) 108 (resting 50 114.0) 1425619658)
 (trackpoint (GPS 19.4907107 -99.2410833) 106 (resting 50 114.0) 1425619662))
\end{verbatim}

\item $\textbf{total-distance}$ - Dada una lista de \verb;trackpoint;s, regresar la distancia total recorrida, \textit{i.e.}
\begin{verbatim}
> (define sample (create-trackpoints (take raw-data 100) my-zones))
> (total-distance (create-trackpoints sample my-zones))
0.9509291243812747
> (define trackpoints (create-trackpoints raw-data my-zones))
> (total-distance trackpoints)
5.051934549322941
\end{verbatim}

\item $\textbf{average-hr}$ - Dada una lista de \verb;trackpoint;s, regresar el promedio del ritmo cardiaco, el resultado debe ser un entero \textit{i.e.}
\begin{verbatim}
> (average-hr sample)
134
> (average-hr trackpoints)
150
\end{verbatim}

\item $\textbf{max-hr}$ - Dada una lista de \verb;trackpoint;s, regresar el máximo ritmo cardiaco, el resultado debe ser un entero \textit{i.e.}
\begin{verbatim}
> (max-hr sample)
147
> (max-hr trackpoints)
165
\end{verbatim}

\item $\textbf{collapse-trackpoints}$ - Dada una lista de \verb;trackpoint;s y un epsilon \verb;e;, obtener una nueva lista en que se agrupen los deltas consecutivos dado que se cumpla lo siguiente: la distancia de un trackpoint al otro trackpoint debe ser menor o igual a \verb;e; y los trackpoints deben tener el mismo ritmo cardiaco, \textit{i.e.}
\begin{verbatim}
> (define sample-four (create-trackpoints (take raw-data 4) my-zones))
> sample-four
(list
 (trackpoint (GPS 19.4907258 -99.24101) 104 (resting 50 114.0) 1425619654)
 (trackpoint (GPS 19.4907258 -99.24101) 104 (resting 50 114.0) 1425619655)
 (trackpoint (GPS 19.4907258 -99.24101) 108 (resting 50 114.0) 1425619658)
 (trackpoint (GPS 19.4907107 -99.2410833) 106 (resting 50 114.0) 1425619662))
> (collapse-trackpoints sample-four 0.01)
(list
 (trackpoint (GPS 19.4907258 -99.24101) 104 (resting 50 114.0) 1425619655)
 (trackpoint (GPS 19.4907258 -99.24101) 108 (resting 50 114.0) 1425619658)
 (trackpoint (GPS 19.4907107 -99.2410833) 106 (resting 50 114.0) 1425619662))
\end{verbatim}

\end{itemize}

\textbf{Sección II. Árboles Binarios}
Para las siguientes funciones, utiliza el tipo de dato \verb;BTree;
definido en \verb;practica3-base.rkt;:
\begin{verbatim}
(define-type BTree
  [EmptyBT]
  [BNode (c procedure?)
         (l BTree?)
         (e any?)
         (r BTree?)])
\end{verbatim}

Donde el primer parámetro del constructor de tipo \verb;BNode; es una función de comparación que recibe dos argumentos y regresa un booleano, indicando si el primer argumento es menor que el segundo. Para un árbol de números, la función de comparación sería \verb;<;, para strings sería \verb;string<?;. \\
    
En esta práctica utilizaremos abreviaciones para los constructores de tipos:

\begin{enumerate}
\item \verb;ebt; es lo mismo que \verb;(EmptyBT);
\item \verb;(bnn ebt 1 ebt); es lo mismo que \verb;(BNode < (EmptyBT) 1 (EmptyBT));
\item \verb;(bns ebt ``hello'' ebt); es lo mismo que \verb;(BNode string<? (EmptyBT) ``hello'' (EmptyBT));
\end{enumerate}

Por último, cuentas con la función \verb;printBT; que recibe un árbol de tipo \verb;BTree; y regresa su representación gráfica, para que puedas depurar tu código: \\

\jimage{0.9}{printbt1.png}

\begin{itemize}
\item \texttt{ninBT} - Dado un árbol de tipo \verb;BTree;, determinar el número de nodos internos que tiene.
\begin{verbatim}
> (ninBT (EmptyBT))
0
> (ninBT (BNode < (BNode < (EmptyBT) 3 (EmptyBT)) 1 (BNode < (EmptyBT) 2 (EmptyBT))))
1
\end{verbatim}

\item \texttt{nlBT} - Dado un árbol de tipo \verb;BTree;, determinar el número de hojas no vacías.
\begin{verbatim}
> (nlBT (EmptyBT))
0
> (nlBT (BNode < (BNode < (EmptyBT) 3 (EmptyBT)) 1 (BNode < (EmptyBT) 2 (EmptyBT))))
2
\end{verbatim}

\item \texttt{nnBT} - Dado un árbol de tipo \verb;BTree;, determinar el número de nodos que tiene. Las hojas vacías no cuentan.
\begin{verbatim}
> (nnBT (EmptyBT))
0
> (nnBT (BNode < (BNode < (EmptyBT) 3 (EmptyBT)) 1 (BNode < (EmptyBT) 2 (EmptyBT))))
3
\end{verbatim}

\item \texttt{mapBT} - Dado una función de aridad 1 y un árbol de tipo \verb;BTree;, aplicar la función sobre todos los valores de los nodos del árbol (las funciones de aridad 1 sólo regresas números).
\begin{verbatim}
> (mapBT add1 (EmptyBT))
(EmptyBT)
> (mapBT add1 (BNode < (EmptyBT) 1 (BNode < (EmptyBT) 2 (EmptyBT))))
(BNode < (EmptyBT) 2 (BNode < (EmptyBT) 3 (EmptyBT)))
> (mapAB (lambda (x) (* x x)) (BNode < (EmptyBT) 3 (BNode < (EmptyBT) 2 (EmptyBT))))
(BNode < (EmptyBT) 9 (BNode < (EmptyBT) 4 (EmptyBT)))
\end{verbatim}
\end{itemize}

\newpage
\item  Ahora, sea \verb;arbol-base;\dots
\begin{verbatim}
(define arbol-base (bns (bns (bns ebt "A" ebt) "B" (bns (bns ebt "C" ebt) "D" (bns ebt "E" ebt))) 
                        "F"
                        (bns ebt "G" (bns (bns ebt "H" ebt) "I" ebt))))
\end{verbatim}

\begin{enumerate}
\item  \textbf{preorderBT} - Dado un árbol de tipo \verb;BTree;, regresar una lista de sus elementos recorridos en preorden.
\begin{verbatim}
> (preorderBT arbol-base)
'("F" "B" "A" "D" "C" "E" "G" "I" "H")
\end{verbatim}

\item  \textbf{inorderBT}  - Dado un árbol de tipo \verb;BTree;, regresar una lista de sus elementos recorridos en inorden.
\begin{verbatim}
> (inorderBT arbol-base)
'("A" "B" "C" "D" "E" "F" "G" "H" "I")
\end{verbatim}

\item \textbf{posorderBT} - Dado un árbol de tipo \verb;BTree;, regresar una lista de sus elementos recorridos en post-orden.
\begin{verbatim}
> (posorderBT arbol-base)
'("A" "C" "E" "D" "B" "H" "I" "G" "F")
\end{verbatim}
   
\end{enumerate}

\jimage{0.8}{orders.png}

\end{document}
