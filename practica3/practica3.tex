\documentclass{article}
\usepackage[left=2cm,right=2cm,top=3cm,bottom=3cm,letterpaper]{geometry}
\usepackage[spanish]{babel}
\usepackage[utf8]{inputenc}

\usepackage{verbatim, array}
\usepackage{hyperref}
\usepackage{amsmath, amsfonts, amssymb}
\usepackage{graphicx}
\usepackage[T1]{fontenc}

\newcommand{\jimage}[2]{\includegraphics[width=#1\textwidth]{#2}\vskip10pt}
\newcommand{\jcimage}[2]{\begin{center}\includegraphics[width=#1\textwidth]{#2}\end{center}\vskip10pt}

\author{Profesora: Karla Ramírez Pulido\\
  Ayudante teoría: Joshua Emmanuel Mendoza Mendieta\\
  Ayudante laboratorio: Héctor Enrique Gómez Morales}
\title{Practica 3 - Data Types Parte 2 de 2}
\date{Fecha de inicio: 20 de febrero de 2015\\
  \textbf{Fecha de entrega: 6 de marzo de 2015}}
\begin{document}
\maketitle
\section{Instrucciones}
En esta práctica se tienen nueve ejercicios, los primeros ocho son
obligatorios siendo solamente el ultimo opcional con valor de un
punto extra. Por lo tanto la calificación máxima que se puede obtener en
esta práctica es 11.

Esta practica se puede entregar en equipos de a lo máximo tres
personas.

Esta práctica debe ser implementada con la variante plai, es decir
su archivo rkt debe tener como primer linea lo siguiente:
\texttt{\#lang plai}. Pueden utilizar sólo la paquetería básica de
\texttt{racket/base} y funciones auxiliares que ustedes definan.

Todos los ejercicios requieren contar con pruebas mediante el uso de
la función \texttt{test}:

\textbf{Cada ejercicio debe contar al menos con CINCO pruebas.}

\section{Ejercicios}
\textbf{Sección I. Heart Rate Zones} Para las siguientes funciones, se utilizaran los tipos de dato \verb;HRZ; y \verb;Time; definidos en \verb;practica3-base.rkt;:
\begin{verbatim}
(define-type HRZ
  [resting (low number?)
           (high number?)]
  [warm-up (low number?)
           (high number?)]
  [fat-burning (low number?)
               (high number?)]
  [aerobic (low number?)
           (high number?)]
  [anaerobic (low number?)
             (high number?)]
  [maximum (low number?)
           (high number?)])

(define-type Time
  [delta (distance number?)
         (lhr number?)
         (hhr number?)
         (zone HRZ?)
         (time number?)])
\end{verbatim}

El tipo de dato \verb;HRZ; representa las zonas de frecuencia cardiaca,
cada zona tiene como parámetros en su constructor el mínimo y máximo ritmo cardiaco
de la zona correspondiente.

El tipo de dato \verb;Time; representa un intervalo de tiempo, se tiene el constructor del tipo \verb;delta; que toma una distancia recorrida, el mínimo y máximo ritmo cardiaco, la zona de frecuencia cardiaca y el tiempo total del intervalo.

\begin{itemize}
\item $\textbf{zones}$ - Dado el ritmo cardiaco de descanso y el máximo ritmo cardiaco de una persona se debe regresar la lista de zonas de frecuencia cardiaca. La formula utilizada comúnmente es:\\
  $range = max - rest$\\
  $zone\ 0\ (resting) = rest (Low\ Bound)$\\
  $zone\ 0\ (resting) = (range * 0.5) - 1 (Upper\ Bound)$\\
  $zone\ (i+1) = rest + range * (0.5 + (0.1 * i))\ (Low\ Bound)$\\
  $zone\ (i+1) = rest + range * (0.5 + (0.1 * (i+1))) - 1\ (Upper\ Bound)$,\\
  donde i = $0..4$, \textit{i.e.}

\begin{verbatim}
> (zones 50 180)
(list
 (resting 50 114.0)
 (warm-up 115.0 127.0)
 (fat-burning 128.0 140.0)
 (aerobic 141.0 153.0)
 (anaerobic 154.0 166.0)
 (maximum 167.0 180))
\end{verbatim}

Para los siguientes ejemplos se define:
\begin{verbatim}
> (define my-zones (zones 50 180))
\end{verbatim}

\item $\textbf{get-zone}$ - Dado un símbolo que es el nombre de una zona y una lista de zonas regresar el tipo de dato correspondiente, \textit{i.e.}
\begin{verbatim}
> (get-zone 'anaerobic my-zones)
(anaerobic 154.0 166.0)
> (get-zone 'maximum my-zones)
(maximum 167.0 180)
\end{verbatim}

\item $\textbf{bpm->zone}$ - Dado una lista de frecuencias cardiacas y una lista de zonas de frecuencia cardiaca regresar una lista de zonas por cada frecuencia cardiaca en la lista original, \textit{i.e.}
\begin{verbatim}
> (bpm->zone empty my-zones)
'()
> (bpm->zone '(50 60) my-zones) 
(list (resting 50 114.0) (resting 50 114.0))
> (bpm->zone '(140 141) my-zones) 
(list (fat-burning 128.0 140.0) (aerobic 141.0 153.0))
\end{verbatim}

\item $\textbf{collect-deltas}$ - Dado una lista de deltas, obtener una nueva lista en que se agrupen las deltas consecutivas que compartan la misma zona. Al agruparse se suman las distancias y tiempos, mientras que se toman los mínimos y máximos de las frecuencias cardiacas del grupo, \textit{i.e.}
\begin{verbatim}
> (collect-deltas (list (delta 10 132 138 mfat 5) 
                        (delta 10 136 139 mfat 8)))
(list (delta 20 132 139 (fat-burning 128.0 140.0) 13))
> (collect-deltas (list (delta 13 134 138 mfat 5) 
                        (delta 14 136 138 mfat 5)
                        (delta 17 141 142 maero 5)
                        (delta 17 141 143 maero 5)))
(list (delta 27 134 138 (fat-burning 128.0 140.0) 10) (delta 34 141 143 (aerobic 141.0 153.0) 10))
\end{verbatim}
\end{itemize}

\textbf{Sección II. Árboles Binarios}
Para las siguientes funciones, utiliza el tipo de dato \verb;BTree;
definido en \verb;practica3-base.rkt;:
\begin{verbatim}
(define-type BTree
  [EmptyBT]
  [BNode (c procedure?)
         (l BTree?)
         (e any?)
         (r BTree?)])
\end{verbatim}

Donde el primer parámetro del constructor de tipo \verb;BNode; es una función de comparación que recibe dos argumentos y regresa un booleano, indicando si el primer argumento es menor que el segundo. Para un árbol de números, la función de comparación sería \verb;<;, para strings sería \verb;string<?;. \\
    
En esta práctica utilizaremos abreviaciones para los constructores de tipos:

\begin{enumerate}
\item \verb;ebt; es lo mismo que \verb;(EmptyBT);
\item \verb;(bnn ebt 1 ebt); es lo mismo que \verb;(BNode < (EmptyBT) 1 (EmptyBT));
\item \verb;(bns ebt ``hello'' ebt); es lo mismo que \verb;(BNode string<? (EmptyBT) ``hello'' (EmptyBT));
\end{enumerate}

Por último, cuentas con la función \verb;printBT; que recibe un árbol de tipo \verb;BTree; y regresa su representación gráfica, para que puedas depurar tu código: \\

\jimage{0.9}{imgs/printbt1.png}

\newpage
\begin{itemize}
\item \texttt{ninBT} - Dado un árbol de tipo \verb;BTree;, determinar el número de nodos internos que tiene.
\begin{verbatim}
> (ninBT (EmptyBT))
0
> (ninBT (BNode < (BNode < (EmptyBT) 3 (EmptyBT)) 1 (BNode < (EmptyBT) 2 (EmptyBT))))
1
\end{verbatim}

\item \texttt{nlBT} - Dado un árbol de tipo \verb;BTree;, determinar el número de hojas no vacías.
\begin{verbatim}
> (nlBT (EmptyBT))
0
> (nlBT (BNode < (BNode < (EmptyBT) 3 (EmptyBT)) 1 (BNode < (EmptyBT) 2 (EmptyBT))))
2
\end{verbatim}

\item \texttt{nnBT} - Dado un árbol de tipo \verb;BTree;, determinar el número de nodos que tiene. Las hojas vacías no cuentan.
\begin{verbatim}
> (nnBT (EmptyBT))
0
> (nnBT (BNode < (BNode < (EmptyBT) 3 (EmptyBT)) 1 (BNode < (EmptyBT) 2 (EmptyBT))))
3
\end{verbatim}

\item \texttt{mapBT} - Dado una función de aridad 1 y un árbol de tipo \verb;BTree;, aplicar la función sobre todos los valores de los nodos del árbol (las funciones de aridad 1 sólo regresas números).
\begin{verbatim}
> (mapBT add1 (EmptyBT))
(EmptyBT)
> (mapBT add1 (BNode < (EmptyBT) 1 (BNode < (EmptyBT) 2 (EmptyBT))))
(BNode < (EmptyBT) 2 (BNode < (EmptyBT) 3 (EmptyBT)))
> (mapAB (lambda (x) (* x x)) (BNode < (EmptyBT) 3 (BNode < (EmptyBT) 2 (EmptyBT))))
(BNode < (EmptyBT) 9 (BNode < (EmptyBT) 4 (EmptyBT)))
\end{verbatim}
\end{itemize}

\newpage
\textbf{Punto extra:}
Ahora, sea \verb;arbol-base;\dots
\begin{verbatim}
(define arbol-base (bns (bns (bns ebt "A" ebt) "B" (bns (bns ebt "C" ebt) "D" (bns ebt "E" ebt))) 
                        "F"
                        (bns ebt "G" (bns (bns ebt "H" ebt) "I" ebt))))
\end{verbatim}

\begin{enumerate}
\item  \textbf{preorderBT} - Dado un árbol de tipo \verb;BTree;, regresar una lista de sus elementos recorridos en preorden.
\begin{verbatim}
> (preorderBT arbol-base)
'("F" "B" "A" "D" "C" "E" "G" "I" "H")
\end{verbatim}

\item  \textbf{inorderBT}  - Dado un árbol de tipo \verb;BTree;, regresar una lista de sus elementos recorridos en inorden.
\begin{verbatim}
> (inorderBT arbol-base)
'("A" "B" "C" "D" "E" "F" "G" "H" "I")
\end{verbatim}

\item \textbf{posorderBT} - Dado un árbol de tipo \verb;BTree;, regresar una lista de sus elementos recorridos en post-orden.
\begin{verbatim}
> (posorderBT arbol-base)
'("A" "C" "E" "D" "B" "H" "I" "G" "F")
\end{verbatim}
   
\end{enumerate}

\jimage{0.8}{imgs/orders.png}

\end{document}
