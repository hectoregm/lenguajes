\documentclass{article}
\usepackage[left=2cm,right=2cm,top=3cm,bottom=3cm,letterpaper]{geometry}
\usepackage[spanish]{babel}
\usepackage[utf8]{inputenc}

\usepackage{verbatim, array}
\usepackage{hyperref}
\usepackage{amsmath, amsfonts, amssymb}
\usepackage{graphicx}
\usepackage[T1]{fontenc}

\newcommand{\gradeone}{(\textbf{1pt}) }
\newcommand{\grade}[1]{(\textbf{#1pts}) }
\newcommand{\jimage}[2]{\includegraphics[width=#1\textwidth]{#2}\vskip10pt}
\newcommand{\jcimage}[2]{\begin{center}\includegraphics[width=#1\textwidth]{#2}\end{center}\vskip10pt}

\author{Profesora: Karla Ramírez Pulido\\
  Ayudante teoría: Joshua Emmanuel Mendoza Mendieta\\
  Ayudante laboratorio: Héctor Enrique Gómez Morales}
\title{Practica 4 - Interprete FAE}
\date{Fecha de inicio: 27 de febrero de 2015\\
  \textbf{Fecha de entrega: 20 de marzo de 2015}}
\begin{document}
\maketitle
\section{Instrucciones}

Para esta práctica, requerirás tomar como base el archivo \texttt{p4-base.rkt} e implementar las funciones que se solicitan.

El objetivo de esta práctica es hacer un intérprete del lenguaje \texttt{FAE} (Function, Arithmetic Expression) con \emph{ambientes}. Se tendrán dos sintaxis, la primera \texttt{FAES} es una sintaxis que se define explícitamente las expresiones \texttt{with}, y la sintaxis \texttt{FAE} en las que solo se tienen las operaciones aritméticas, definición de funciones y aplicación de funciones.

Usaremos la definición y aplicación de funciones para implementar la funcionalidad del \texttt{with}, dado que:
\begin{verbatim}
{with {var named} body}

\end{verbatim}

lo reemplazamos con

\begin{verbatim}
{{fun {var} body}
  named}
\end{verbatim}

Es decir las expresiones \texttt{with} son \emph{syntatic sugar}. Para realizar esto se definirá una función \texttt{desugar} que tomara una expresión en sintaxis \texttt{FAES} a una expresión en sintaxis \texttt{FAE}. Para luego hacer el interprete del árbol de sintaxis abstracta de \texttt{FAE} que debe ser de alcance estático por medio de ambientes.

\begin{verbatim}
<binop>::= +
         | -
         | *
         | /

<FAES>::= <num>
         | <id>
         | {<binop> <FAES> <FAES>}
         | {with {{<id> <FAES>}+} <FAES>}
         | {fun {<id>+} <FAES>+}
         | {<FAES> <FAES>+}

<FAE>::= <num>
         | <id>
         | {<binop> <FAE> <FAE>}
         | {fun {<id>+} <FAE>+}
         | {<FAE> <FAE>+}
\end{verbatim}

\begin{center}
\framebox[1.1\width]{\textbf{lexer (source $\rightarrow$ listof tokens)}} $\rightarrow$
\framebox[1.1\width]{\textbf{parser (listof tokens $\rightarrow$ FAES)}} $\rightarrow$
\framebox[1.1\width]{\textbf{desugar (FAES $\rightarrow$ FAE))}} $\rightarrow$
\framebox[1.1\width]{\textbf{interp (FAE $\rightarrow$ FAE-Value))}}
\end{center}

Esta práctica se puede entregar en equipos de a lo más tres
personas.

Esta práctica debe ser implementada con la variante plai, es decir
su archivo rkt debe tener como primer linea lo siguiente:
\texttt{\#lang plai}.

Todos los ejercicios requieren contar con pruebas mediante el uso de
la función \texttt{test}:

\section{Ejercicios}

\begin{enumerate}

\item \grade{3} \textbf{desugar} Define una función que toma una expresión en sintaxis \texttt{FAES} y que regresa una expresión en sintaxis \texttt{FAE}.

\item \grade{3} \textbf{multi-param} Adecua el \texttt{interp} de tal manera que las funciones acepten una lista de parámetros y que las aplicaciones de funciones sean de múltiples argumentos.

\item \grade{4} \textbf{interp} En base al código de \texttt{p4-base.rkt}, implementar el intérprete, dado un árbol de sintaxis abstracta, evaluar y regresar un valor de tipo \texttt{FAE-Value} que considera las variantes numV y closureV. closureV es el constructor de tipo que recibe un parámetro, tiene un cuerpo de tipo \texttt{FAE} y un ambiente de tipo Env. Env considera las variantes de tipo mtSub y aSub, donde el constructor aSub recibe un símbolo, un valor de tipo \texttt{FAE-Value} y otro ambiente Env.

\end{enumerate}

\textbf{Punto extra:}
\begin{enumerate}
\item \textbf{lexer} Dada una cadena de caracteres, generar el código de sintaxis concreta del lenguaje FAE. Nota que un lexer sencillo sólo genera \emph{tokens} del código fuente que se lee directamente como cadenas de caracteres. Nuestro \emph{lexer} a diferencia, realizará el balanceo de paréntesis generando listas anidadas acorde a los caracteres \texttt{\{} y \texttt{\}} que se vayan presentando.
\begin{verbatim}
> (lexer "{with {{x 1} {y 2} {f {fun {x} {+ x 1}}}} {f y}}")
'(with ((x 1) (y 2) (f (fun (x) (+ x 1)))) (f y))
\end{verbatim}
Observación. El código de sintaxis concreta sólo son listas, símbolos y números. Ésta debe ser una entrada válida al parser si la cadena de caracteres es correcta.
\end{enumerate}

\end{document}
