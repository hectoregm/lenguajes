\documentclass{article}
\usepackage[left=2cm,right=2cm,top=3cm,bottom=3cm,letterpaper]{geometry}
\usepackage[spanish]{babel}
\usepackage[utf8]{inputenc}

\usepackage{verbatim, array}
\usepackage{hyperref}
\usepackage{amsmath, amsfonts, amssymb}
\usepackage{graphicx}
\usepackage[T1]{fontenc}

\newcommand{\gradeone}{(\textbf{1pt}) }
\newcommand{\grade}[1]{(\textbf{#1pts}) }
\newcommand{\jimage}[2]{\includegraphics[width=#1\textwidth]{#2}\vskip10pt}
\newcommand{\jcimage}[2]{\begin{center}\includegraphics[width=#1\textwidth]{#2}\end{center}\vskip10pt}

\author{Profesora: Karla Ramírez Pulido \and
  Ayudante: Héctor Enrique Gómez Morales}
\title{Practica Reposición - Recolección de Basura}
\date{Fecha de inicio: 11 de diciembre de 2015\\
  \textbf{Fecha de entrega: 4 de enero de 2016}}
\begin{document}
\maketitle
\section{Instrucciones}
Implementarás un recolector de basura, a escoger \textbf{marcado & barrido} o \textbf{detente & copia}.
Como vimos, la recolección de basura involucra comenzar desde el \textit{conjunto raíz} de referencias,
que residen en las variables locales y los \textit{stack frames}, y buscar los valores alcanzables en el heap.
Para esta práctica incluimos código de soporte para que tu recolector tenga acceso al conjunto raíz de un
programa en ejecución, de tal forma que solo tendrás que concentrarte en la implementación de los algoritmos de
recolección de basura.

El codigo de soporte esta incluido en el archivo zip que contiene este documento, también esta disponible
en: \textbf{http://valhalla.fciencias.unam.mx/lenguajes/gc-support.tgz}

\section*{GC Collector Racket}
En tu recolector deberás utilizar el lenguaje \textbf{#lang plai/collector} al inicio del archivo
para tener a tu disposicion los siguientes procedimientos:

\begin{itemize}
\item \verb;(get-root-set id ...);: produce una lista que representa el conjunto actual de raíces, incluyendo
  una entrada por cada \verb;id; adicional.
\item \verb;(root-name root);: dada una raíz, regresa su nombre como un símbolo. Si la raíz corresponde a una
  variable del programa, el nombre sera el mismo. En otro caso, sera tmpXXX. Este procedimiento te ayudara a
  depurar tu recolector.
\item \verb;(read-root root);: dada una raíz, regresa la localidad a la que hace referencia en el momento
  actual.
\item \verb;(set-root! root new-loc);: dada una raíz y una localidad, actualiza la raíz para hacer referencia
  a una nueva localidad.
\item \verb;(procedure-roots proc);: dado un closure almacenado en el heap, regresa el conjunto raíz que es
  alcanzable desde el ambiente del closure, si el \verb;proc; no es alcanzable, regresa la lista vacía.
\end{itemize}

\section*{collector.ss}
Incluimos un \textbf{collector} trivial que no realiza recolección de basura (i.e., cuando se acaba la memoria
simplemente se detiene y marca un error). Tendrías que realizar tu propio modulo -para el algoritmo de recolección
que escogiste- que implementen la misma interfaz. La interfaz es como sigue:

\begin{itemize}
\item \verb;gc:alloc-flat : num U sym U bool U empty -> loc;: Este procedimiento debería almacenar un valor
  plano de Racket (number, symbol, boolean, empty list) en el heap, regresando su localidad (un numero). El valor
  debería ocupar una celda en el heap, aunque puedes usar espacio adicional para almacenar una etiqueta, etc.
  Puedes pre-almacenar constantes comunes (e.g., empty list). De ser necesario, este procedimiento debe
  realizar recolección de basura, si después de esto, el espacio sigue siendo insuficiente, se debería lanzar
  un error.
\item \verb;init-allocator : -> void;: Este procedimiento es el primero que ejecuta un \textbf{mutator}. Aquí
  puedes poner cualquier condición inicial para tu \textbf{collector}, en la implementación que se entrega no se
  hace nada.
\item \verb;gc-flat? : loc -> boolean;: Este procedimiento debería regresar \verb;true; si el contenido de la
  localidad dada es de tipo "flat". En otro caso, debería regresar \verb;false;, nunca deberá regresar un error.
\item \verb;gc:deref : loc -> num U sym U bool U empty;: Dada una localidad de un valor plano de Racket,
  este procedimiento regresaría ese valor; si la localidad apunta a una celda \verb;cons;, esta función debería
  lanzar un error.
\item \verb;gc:cons : loc loc -> loc;: Este procedimiento debería almacenar una celda cons en el heap, cuyos
  \verb;first; y \verb;rest; son los argumentos dados. Estos campos deben ser almacenados en celdas diferentes
  del heap, y puedes utilizar espacio adicional para una etiqueta, etc. De ser necesario, como con
  \verb;alloc-flat;, este procedimiento puede realizar recolección de basura. Si después de esto, el espacio sigue siendo insuficiente, se debería lanzar un error.
\item \verb;gc:cons? : loc -> boolean;: Debería regresar \verb;true; si la localidad se refiere a una celda
  \verb;cons;
\item \verb;gc:first : loc -> loc;: Si la localidad dada se refiere a una celda \verb;cons;, debería regresar
  el campo \verb;first;. En otro caso, debería lanzar un error.
\item \verb;gc:rest : loc -> loc;: Si la localidad dada se refiere a una celda \verb;rest;. En otro caso, debería
  lanzar un error.
\end{itemize}

\section*{GC Mutator Racket (mutator.ss)}
Este modulo define un lenguaje en el cual puedes implementar aplicaciones de prueba (\textbf{mutators}) para
tus recolectores. Un programa escrito en este lenguaje obtendrá sus raíces a través de \verb;get-root-set;,
almacenara sus datos en tu heap, y usara las versiones de las primitivas definidas en tu \textbf{collector}.
(Cuando se hace referencias a procedimientos como \verb;cons; o \verb;symbol?;, las llamadas son automáticamente
redireccionadas a \verb;gc:cons; y \verb;gc:symbol?; definidas en tu \textbf{collector}).
El tamaño del heap que se utiliza se define mediante el procedimiento \verb;allocator-setup;, el cual recibe
el nombre del archivo donde se encuentra tu \textbf{collector} y el tamaño del heap que se utiliza.

\section*{Notas}
Durante la recolección, necesitaras usar \verb;get-root-set; para obtener el conjunto de raíces actual. En
\verb;gc:alloc-flat;, no deberías listar ningún identificador adicional. Sin embargo, en \verb;gc:cons;, debe
pasar los argumentos (los campos \verb;first; y \verb;rest;), ya que son raíces (Si no los listas, sera necesario
que los trates de manera especial y los actualices con \verb;set!;).

En el recolector \textbf{marcado & barrido} debes mantener una lista libre en el heap. Es decir, no deberías
usar una estructura de datos auxiliar; en vez de eso, utiliza el espacio disponible en el heap para mantener
registro de la lista libre. Puedes usar una caja extra ("\textbf{register}") para apuntar al inicio de la lista
libre. Es posible que necesites ajustar tu almacenamiento para tener suficiente espacio para guardar apuntadores
a listas libres!.

Probablemente quieras utilizar números para representar localidades, pero puedes utilizar otra representación
si esta facilita tu implementación.

Otros consejos:
\begin{itemize}
\item Prueba tu programa con tamaños de heap pequeños, para que la recolección de basura se realice
  frecuentemente.
\item No necesitas compactar la memoria en \textbf{marcado & barrido}.
\item Puede ser conveniente usar el constructor \textbf{set!} de Racket, el cual permite mutar una variable
  sin usar cajas. Te recomendamos utilizar esta característica en un caso: al intercambiar los semi-espacios
  en el recolector \textbf{detente & copia}. Esto te evitara el problema de "destapar" el heap cada vez que lo
  uses.
\end{itemize}

\section*{FAQ}

\begin{itemize}
\item \verb;get-root-set; parece regresar raíces duplicadas. ¿ Es un bug en el código de soporte ? No,
  ese es el comportamiento que se quiere. Las raíces vienen de recorrer el stack, y es posible que muchos frames
  hagan referencia a la misma variable. Es mas, variables diferentes pueden hacer referencia a la misma localidad
  en el heap. Necesitas actualizar \textbf{todos} ellos en tu recolector.
\item Seria mucho mas fácil si pudiera hacer que todos los objetos consumieran la misma cantidad de espacio. ¿ Eso seria aceptable ? No! Hemos sido extremadamente tolerantes en que solo manejes \textit{dos} tamaños. Reducirlo
  a un tamaño haría que el problema no fuera realista y por el contrario fuera trivial.
\end{itemize}

\end{document}
