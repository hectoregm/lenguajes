\documentclass{article}
\usepackage[%
left=2cm,% left margin
right=2cm,% right margin
top=3cm, % top margin
bottom=3cm,% bottom margin
letterpaper% other options: a0paper, a1paper, a2paper, a3paper, a4paper, a5paper, a6paper, and many more.
]{geometry}
\usepackage[spanish]{babel}
\usepackage[utf8]{inputenc}
\author{Profesora: Karla Ramírez Pulido\\
  Ayudante teoría: Joshua Emmanuel Mendoza Mendieta\\
  Ayudante laboratorio: Héctor Enrique Gómez Morales}
\title{Practica 0 - Principios de Racket y Recursión}
\date{Fecha de inicio: 7 de febrero de 2015\\
  \textbf{Fecha de entrega: 21 de febrero de 2015}}
\begin{document}
\maketitle
\section{Instrucciones}
En esta practica se tienen once ejercicios, los primeros diez son
obligatorios siendo solamente el ultimo opcional con valor de un punto
extra. Por lo tanto la calificacion maxima que se puede obtener en
esta practica es 11.
\section{Ejercicios}
Define las funciones que se te piden. Puedes crear y utilizar
funciones auxiliares. No puedes utilizar funciones de Racket que
resuelvan directamente los ejercicios.
\begin{enumerate}
\item{$\textbf{zip}$ - Dadas dos listas, regresar una lista cuyos elementos son
  listas de tamaños dos, tal que par la i-ésima lista, el primer
  elemento es el i-ésimo de la primera lista original y el segundo
  elemento es el i-ésimo de la segunda lista original, si una lista es
de menor tamaño que la otra, la lista resultante es del tamaño de la
menor, y si una de las listas es vacía, regresar una lista vacia,
\textit{i.e.}}
\begin{verbatim}
> (zip ’(1 2) ’(3 4))
’((1 3) (2 4))
> (zip ’(1 2 3) ’())
’()
> (zip ’() ’(4 5 6))
’()
> (zip ’(8 9) ’(3 2 1 4))
’((8 3) (9 2))
> (zip ’(8 9 1 2) ’(3 4))
’((8 3) (9 4))
\end{verbatim}
\end{enumerate}
\end{document}
